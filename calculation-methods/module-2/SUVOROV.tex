\documentclass[12pt,a4paper]{article}
\usepackage[T2A,T1]{fontenc}
\usepackage[utf8]{inputenc}
\usepackage[ukrainian]{babel}
\usepackage{amsmath}

\hoffset=-0.5cm       
\voffset=-1.5cm        
\oddsidemargin=0.1cm  
\evensidemargin=0.1cm   
\topmargin=0.1cm      
\headheight=0.5cm     
\footskip=1cm         
\headsep=0.3cm        
\textwidth=17cm       
\textheight=25.5cm    

\begin{document}
\thispagestyle{empty}

 \begin{center}
 \large
 {Міністерство освіти і науки України}
 \\{Львівський національний університет імені Івана Франка}
 \\{Факультет електроніки та комп'ютрених наук}
 \end{center}

 \vspace{2cm}

 \begin{flushright}
 \large Кафедра систмного проектування
 \end{flushright}

 \vfill

 \begin{center}
 {\Huge{Реферат}}\\
 {\large{на тему}:}
 \end{center}

 \vspace{0.5cm}

 \begin{center}\Large
 \textbf{\emph{``LU-розклад матриці. Ітераційні методи уточнення розв'язків системи лінійних алгебраїчних рівнянь''}}
 \end{center}

 \vfill


 \begin{flushleft}
 \hskip10cm \large
 {Виконав:}
 \end{flushleft}

 \begin{flushleft}
 \hskip10cm \large
 {студент групи ФеП-31}
 \end{flushleft}

 \begin{flushleft}
 \hskip10cm \large
 {Суворов О.А.}
 \end{flushleft}

 \begin{flushleft}
 \hskip10cm \large
 {Перевірив}:
 \end{flushleft}

 \begin{flushleft}
 \hskip10cm \large
 {проф. Бугрій О.М.}
 \end{flushleft}

 \vfill

 \begin{center}
 \large
 {Львів - 2023}
 \end{center}

 \newpage
 \section{Теоретичні відомості про LU-розклад матриці}
 \hspace{1 cm} \(LU\)-розклад матриці - це метод обчислення, який дозволяє представляти матрицю у вигляді добутку нижньої трикутної матриці \(L\) та верхньої трикутної матриці \(U\). Цей метод заснований на факторизації матриці та має застосування в різних галузях, таких як матричний аналіз, геометричні проблеми та інше.

 Основні теми, де використовується \(LU\)-розклад:
 \begin{itemize}
    \item Представлення матриці у вигляді добутку \(L\) та \(U\).
    \item Застосування \(LU\)-розкладу для розв'язку СЛАР.
    \item Використовується для обчислення оберненої матриці.
    \item Детермінант може бути обчислений через добуток детермінанту матриці \(L\) та детермінанту матриці \(U\).
 \end{itemize}

 \section*{Алгоритм розкладу LU-матриці}
 \hspace{1 cm} Перед самим алгоритмом необхідно пояснити (описати) обов'язкові визначення, які будуть використовуватись для розв'язку матриці даним способом:
 \begin{itemize}
    \item \(L\)-матриця: нижня трикутна матриця, в якій всі діагональні елементи рівні одиниці.
    \item \(U\)-матриця: верхня трикутна матриця, в якій всі діагональні елементи також рівні одиниці.
    \item \(A\)-матриця: матриця добутку двох матриць (\(L\) та \(U\)).
    \item Означення добутку двох матриць: \(a_{ik} = \sum_{j=1}^{n} l_{ij} u_{jk}\)
 \end{itemize}

 \hspace{1 cm} Алгоритм LU-розкладу - є доволі простим методом, тому складається він не з багатьох наступних кроків:
 \begin{enumerate}
    \item Задаємо дві порожні матриці, тобто всі значення матриць 0.
    \item Заповнюємо діагональні елеменети в матриці \(U\) одиницями.
    \item Обраховуємо почергово значення елементів \(k\) стовпця матриці \(L\), а також \(k\) рядка матриці \(U\) за даними формулами (спочатку елементи стовпчика матриці \(L\), а потім елементи рядка \(U\) ):
    \[ l_{ik} = a_{ik} - \sum_{j=1}^{k-1} l_{ij} u_{jk}, i \geq k\]
    \[ u_{ki} = \frac{1}{l_{kk}} [a_{ki} - \sum_{j=1}^{k-1} l_{kj} u_{ji}], i > k \]
 \end{enumerate}

 \section{Теоретичні відомості по ітерацiйним методам уточнення розв’язкiв системи лiнiйних алгебраїчних рiвнянь}
 \hspace{1 cm} Ітераційні методи уточнення розв’язку СЛАР – це методи, які дозволяють наближено розв’язати систему лінійних алгебраїчних рівнянь (СЛАР) шляхом послідовного наближення до розв’язку. Ці методи є ефективними для розв’язання СЛАР великого розміру, коли прямі методи, такі як метод Крамера або метод Гауса, є непридатними.

 \hspace{0.4 cm} Основна ідея ітераційних методів полягає в тому, що вихідне наближення до розв’язку використовується як початкове значення для послідовності функцій, які наближаються до розв’язку. При цьому кожне наступне наближення повинно бути кращим за попереднє.

 \hspace{0.4 cm} Найпопулярніші методи:
 \begin{itemize}
    \item Метод Якобі: \(x_{k + 1} = x_k - L^{-1} (b - A x_k)\), де \(x_k\) - наближення до розв’язку в \(k\)-ій ітерації, \(L\) – нижня трикутна матриця, отримана з матриці \(A\), \(b\) – вектор правих частин, \(x\) – вектор невідомих.
    \item Метод Гауса-Зейделя: даний метод є модифікацією методу Якобі, в якому на кожній ітерації використовується не тільки нижня трикутна матриця, а й верхня трикутна матриця, отримана з матриці \(A\). Формула для методу Гауса-Зейделя має такий вигляд: \(x_{k + 1} = x_k - (L^{-1} + U^{-1}) (b - A x_k) \)
 \end{itemize}

 \hspace{0.4 cm} Для того, щоб визначити, який ітераційний метод слід використовувати для розв’язання конкретної СЛАР, необхідно враховувати такі фактори:
 \begin{itemize}
    \item Розмір СЛАР: Ітераційні методи ефективніші для розв’язання СЛАР великого розміру.
    \item Структура матриці \(A\): Ітераційні методи, які використовують нижню і верхню трикутні матриці, ефективніші для матриць, які мають такий вид.
 \end{itemize}

 \section*{Алгоритм розв'язку СЛАР використовуючи вже розкладеної LU-матриці}
 \hspace{1 cm} Якщо задана система лінійних рівнянянь по типу \(A_{x} = LU_{x} b\) та задано \(A\) і \(b\), тоді кроки для розв'язку даного рівняння будуть виглядати ось так:
 \begin{enumerate}
    \item Розв'язати рівняння \(L_{y} = b\) та знайти y.
    \item Розв'язати рівняння \(U_{x} = y\) та знайти x.
 \end{enumerate}
 \hspace{1 cm} Тепер пройдемося по даному методу трохи детальніше, а саме як саме обрахувати значення для \(x\) та \(y\).
 \[ y_1 = \dfrac{b_1}l_{11} \] 
 \[ y_k = \dfrac{1}{L_{kk}} (b_k - \sum_{j=1}^{k-1} L_{kj} y_{j}), k=2, ..., n \]
 \[ x_n = y_n \]
 \[ x_k = (y_k - \sum_{j=k+1}^{n} U_{kj} x_j), k=n-1, ..., 1 \]

 \section*{Алгоритм розв'язку СЛАР використовуючи метод Якобі}
 \hspace{1 cm} Метод Якобі характеризується сумою трьох матриць: \(A = A^{+} + D + A^{-}\), де \(A^{+}\) - нижня трикутна матриця, D - матриця діагональна, \(A^{-}\) - верхньотрикутна матриця з нульовими елементами по діагоналі. Вихідну систему рівнянь можна подати у вигляді:
 \[ DX = -(A^{+} + A^{-}) X + f\]
 \[X = -D^{-1} (A^{+} + A^{-}) X + D^{-1} f\]
 \[X^{k+1} = -D^{-1} (A^{+} + A^{-}) X^{k} + D^{-1} f\]
 \begin{center}
    або в канонічній формі
    \( X^{k+1} = CX^{k} + d \), де \(C = -D^{-1} (A^{+} + A^{-})\)
    Отже, метод Якобі в розгорнутому вигляді буде ось таким:
    \[ x_i^{k+1} = - \sum_{j=1, j \ne 1}^{n} \dfrac{a_{ij}}{a_{ij}} x_j^k + \frac{f_i}{a_{ii}}\]
 \end{center}

 \section*{Алгоритм розв'язку СЛАР використовуючи метод Гауса-Зейделя}
 \hspace{1 cm} Головна відмінність даного методу, що ми записуємо матрицю іншим виглядом. В цьому методі записуємо матрицю під виглядом суми 3-ох матриць, а саме \(A = A^+ + D +A^- \), де всі параметри були описані у минулій секції даного реферату. Представимо вихідну систему у вигляді:
 \[ (A^+ + D) X = -A^- X +f \]
 \[ (A^+ + D) X^k+1 = -A^- X^k +f \]
 Щоб отримати достатнью умову збіжності методу Зейделя, треба записати його в канонічній формі i методу простої iтерації:
 \[ X^{k+1} = - (A^+ + D)^{-1} A^- X^k + (A^+ + D)^{-1} f \]
 Метод Гауса-Зейделя в розгорнутій формі:
 \[ x_i^{k+1} = \frac{f_i}{a_{ii}} - \sum_{j=1}^{i-1} \frac{a_{ij}}{a_{ii}} x_j^{k+1} - \sum_{j=i+1}^{n} \frac{a_{ij}}{a_{ii}} x_j^k \]

 \section{Розв'язання СЛАР методом Якобі з використанням LU-розкладу}

Розглянемо систему лінійних рівнянь:
\[
\begin{bmatrix}
    4 & -1 & 0 \\
    -2 & 5 & 1 \\
    1 & 1 & 3
\end{bmatrix}
\begin{bmatrix}
    x_1 \\
    x_2 \\
    x_3
\end{bmatrix}
=
\begin{bmatrix}
    8 \\
    -1 \\
    4
\end{bmatrix}
\]

Для застосування методу Якобі розкладемо матрицю коефіцієнтів \(A\) на суму нижньої трикутної матриці \(L\), верхньої трикутної матриці \(U\) і діагональної матриці \(D\):
\[
A =
\begin{bmatrix}
    4 & -1 & 0 \\
    -2 & 5 & 1 \\
    1 & 1 & 3
\end{bmatrix}
=
\begin{bmatrix}
    0 & 0 & 0 \\
    -2 & 0 & 0 \\
    1 & 1 & 0
\end{bmatrix}
+
\begin{bmatrix}
    4 & -1 & 0 \\
    0 & 5 & 1 \\
    0 & 0 & 3
\end{bmatrix}
\]

Запишемо новий вигляд системи, де \(L\), \(U\) та \(D\) представлені як окремі матриці:
\[
(D + L) x^{(k+1)} = U x^{(k)} + b
\]

Розв'язуємо отриману систему для кожного кроку методу Якобі. Початковий набір значень \(x^{(0)}\) може бути вибраний довільно.

\[
\begin{bmatrix}
    4 & 0 & 0 \\
    -2 & 5 & 0 \\
    1 & 1 & 3
\end{bmatrix}
\begin{bmatrix}
    x_1^{(k+1)} \\
    x_2^{(k+1)} \\
    x_3^{(k+1)}
\end{bmatrix}
=
\begin{bmatrix}
    0 & -1 & 0 \\
    0 & 0 & -1 \\
    0 & 0 & 0
\end{bmatrix}
\begin{bmatrix}
    x_1^{(k)} \\
    x_2^{(k)} \\
    x_3^{(k)}
\end{bmatrix}
+
\begin{bmatrix}
    0 \\
    1 \\
    4
\end{bmatrix}
\]

Знову записуємо систему вигляді \(Ax = b\) і розв'язуємо її для кожного кроку методу Якобі.

\section{Висновок}
\hspace{1 cm} Розглянуті методи обчислень, а саме LU-розклад матриці та ітераційні методи уточнення розв'язків систем лінійних алгебраїчних рівнянь, відіграють важливу роль у сучасній обчислювальній математиці та інженерних дисциплінах.

\hspace{0.5 cm}LU-розклад матриці є потужним інструментом для розв'язання систем лінійних рівнянь, дозволяючи представити матрицю як добуток нижньотрикутної та верхньотрикутної матриці. Це полегшує обчислення та зменшує витрати часу.

\hspace{0.5 cm} Щодо ітераційних методів, вони виявляються ефективними при розв'язанні великих та розріджених систем рівнянь. Застосування цих методів дозволяє зблизитися до точного розв'язку системи, покращуючи збіжність та зменшуючи обчислювальні витрати.

\hspace{0.5 cm} У контексті навчання комп'ютерних наук, вивчення цих методів є важливим кроком для розуміння основ оптимізації обчислень та побудови ефективних алгоритмів. Результати застосування цих методів можуть мати широкий спектр застосувань в різних галузях від інженерії до науки про дані.

\begin{thebibliography}{99}
\bibitem{Бугрій О.М.}
Бугрій О.М.
\emph{Лекція по СЛР}

\bibitem{Wikipedia}
Wikipedia
\emph{LU-розклад матриці}

\bibitem{Wikipedia}
Wikipedia
\emph{Метод Якобі}

\bibitem{Wikipedia}
Wikipedia
\emph{Метод Гауса — Зейделя}

\bibitem{author1}
В.М. Глушкова, А.А. Гончарова, В.В. Ремінського
\emph{Чисельні методи}
Сторінки 184-212
\end{thebibliography}

\end{document}
